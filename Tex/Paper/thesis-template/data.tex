\section{Data and Descriptive Evidence} \label{Sec:Data}
The data used in this analysis consists of the data on all full time employed individuals in the A and C samples of the Socio Economic Panel. To slow the ageing and attrition of the sample individuals from later samples were also included when the amount of Old and New Experience or education could still be inferred with reasonable accuracy. These include mainly younger individuals for whom education was finished after reunification and therefore all experience can be assumed to be New experience. In this section first the distribution of gross monthly wages are analysed in East and West Germany across different groups regarding experience and education. This will give some first insights hypothesises regarding the research questions stated above. Afterwards the distribution of these explanatory variables themselves will be analysed, which will be helpful in understanding and interpreting the results in Section \ref{Sec:Results}.
\subsection{Distribution of Wages}
Regarding the overall mean wages in  the East and West German sample one can observe from Figure \ref{fig:MeanWages} that wage levels in the East  started off at a much lower level than in the West. Whereas in the early nineties there was a substantial increase in East German wages relative to the West, this convergence has slowed and there remains a sizeable difference even in the last time period. 
With respect to the distribution of wages across total experience Figure \ref{fig:MeanWagesByTotalExp} shows a similar picture. The wage distribution across overall experience started out very flat in East Germany but then converged towards a similar relative wage structure as in the West over time. This convergence appears at a much higher rate in the nineties and has since slowed down sharply. From this analysis one might expect to see relative returns to total education start off from a significantly lower level the west with regional differences decreasing rapidly in the nineties but then remaining at still somewhat lower levels in later years.
In Figures \ref{fig:MeanWagesByOldExp} and \ref{fig:MeanWagesByNewExp} this analysis is repeated separately for Old and New Experience. Here we see a striking differences between the two kinds of experience. Throughout the time frame there is virtually no difference in mean wages across Old Experience in East Germany. This suggests that experience gathered in the GDR has no effect on the wages post reunification. When comparing this with the picture for West Germany where there are large differences across Old Experience in the early years after reunification, which then disappear over time, this implies, that this difference is probably due to the low quality of human capital obtained through work experience gathered in the socialist economy. Regarding the wage profile across new experience one is faced with the problem that the amount of New Experience an individual has in any given year is limited by the time that has passed since reunification. Therefore early after reunification there are no individuals individuals in the upper groups and the analysis based on Figure \ref{fig:MeanWagesByNewExp} is limited to later years. Although it is not immediately obvious from looking at the graph, the relative differences between the groups is higher in the West throughout the 2000s, a difference that is decreasing only sightly over time.
Regarding the wage profile across education one can see from Figure \ref{fig:MeanWagesByTotalEdu} that education has some effect in East Germany immediately after reunification, which is much lower than in the West. Throughout the whole time frame however the relative wages of the highest education group in East Germany rise steadily and the regional wage distributions seem relatively similar at the end of the time frame. In Figure \ref{fig:MeanWagesByOldEdu} we see an overall relatively similar picture for Old Education which suggests, that the above observations regarding the obsolescence of Old Experience do not apply to education.

\subsection{Distribution of Explanatory Variables}
To better understand and interpret the results in section \ref{Sec:Results} it is useful to take a look at the different regional levels of Experience and Education over time. 
From Figure \ref{fig:MeanExp} one can see that the total level of experience is higher in the West than it is in the East, with the relative difference being particularly high with regards to Old Experience. With respect to education one can see from Figure \ref{fig:MeanEdu}, that while the total levels of education are relatively similar in both regions, people in the East tend to have a higher share of Old Education. Both of these observations can be explained by an overall higher mean age of the East German sample.



\FloatBarrier