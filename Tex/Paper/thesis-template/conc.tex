\section{Conclusions}\label{Sec:Conc}
In the following paragraphs the results of this study will be summarized with regards to the research questions defined in section \ref{Sec:Intro} and the policy implications of these results discussed. Also an outlook will be given into possible ways to extend the research in this study especially with regards to limitations of this study.
\subsection{Research Questions}
Concerning the research questions posed above this study comes to the following conclusions:
\subparagraph*{How do returns to education and experience differ in East and West Germany ?} Overall the differences in relative returns to education are quite small across sampling regions. This also holds when looking at the relative human capital stored in education, which is in fact higher in the East for later periods.  For relative returns to and human capital stored in experience a somewhat different picture emerges. Both are significantly lower in the East throughout the time frame despite higher overall levels of experience in the East German samples.
\subparagraph*{How do these differences develop over time ?}
	Especially regarding the returns to experience there has been a period of rapid decrease in the nineteen-nineties. This trend however has significantly slowed since and there remains a sizeable regional gap relative returns and human capital even 24 years after reunification which shows no trend of disappearing.  
\subparagraph*{How do these differences behave when differentiating between Experience and Education obtained pre- / post-unification?}
	This study was able to confirm the results regarding Experience gathered in the German Democratic Republic from the literature as well as the observations regarding this topic from the exploratory data analysis. This type of experience seems to have no positive effect on wages in the post reunification labour market, and can therefore be said to be obsolete in some sense. In extension to the existing literature this result has been made even clearer when comparing the results to those for experience gathered in the Federal Republic pre unification which does retain value in the first decade after reunification. Therefore this relative lack of returns to Old Experience can not be simply explained by the time passed since reunification but is most likely caused by an inferior quality of work experience gathered under socialism. This fact also offers a possible  explanation for the rapid decreases of differences in the returns to experience in the nineteen-nineties. During this time initially rather high returns to Old Experience in West Germany converged towards zero slowly eliminating this source of regional differences in the wage profile. However the since remaining difference in returns to experience seems to be the part caused by the persisting  difference in evaluations of work experience gathered since reunification. Regarding Education there seems to be no difference between the pre- and post unification and regional differences seem to be rather low for all types and all time periods. 
\subsection{Policy Implications}
The results of this study have implications regarding both the evaluation of past policies pursued since reunification and the design and selection of new policies aimed at reducing economic imbalances between East and West Germany. First it offers a more precise evaluation of policies aimed at providing work experience for struggling workers in East Germany. Some studies have cited the flat wage distribution across experience as evidence against these kind of policies. However since these policies would increase the amount of New Experience, which provides much higher returns, this study offers a more positive view on these kind of policies. Since old individuals have a higher share of work experience gathered pre unification this study also provides evidence that these individuals are especially vulnerable to relative wage losses and might be chosen as the target group of policies mitigating these effects. Another group of individuals especially vulnerable to erosion of human capital are those individuals which have lower levels of education and a higher share of human capital in generally experience, since these people will not profit from the high returns to education in East Germany. Again this group might be of special interest in policy design. Another effect of these high returns to education might be favouring policies that aim at enabling workers in East Germany to update their education levels as opposed to those that aim to provide work experience.
\subsection{Limitations / Future Research}
One of the main Limitations in this study with regards to the interpretation of the results, might be the fact that it offers no insight into wether differences in returns are due to different evaluations of the same human capital in East and West Germany or whether it is due to differences in quality of experience or education gathered in each of the regions. One way to get some insight into this might be analysing the labour market performance of migrants who have gained their experience in one region but earn their wage in the other region. Whether this is possible with the GSOEP data however is questionable since it contains relatively few migrants. Another aspect to consider are different price levels in East and West Germany, for which this study does not account. Adjusting wages to purchasing parity levels might give somewhat different results, which might be more relevant when analysing the economic situation of individuals in East Germany. Another area in which one might be able to improve upon this study is methodology. Since models are fitted on to time periods of several years, each of these estimations is still done on panel data. Regarding the limitations of Pooled OLS in this case and the possibility to apply different estimation methods, further research might be of relevance. Another direction of further research could also be analysing how the results to above questions differ with individuals of different characteristics. In the progress of this study this has been done among others for people with different degrees and has shown some promise.