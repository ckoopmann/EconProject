\section{Results} \label{Sec:Results}
In the following paragraphs, the results obtained through the modelling approach specified in Section \ref{Sec:Method} will be illustrated in a mostly graphical way.
Due to the non-linearity of the experience term in the models specified above the effect of experience on wages would not be immediately obvious from the coefficients. Therefore the returns to experience will be captured through the predicted contribution of 5 years of experience to the log wage. To get this contribution for Total Experience in West Germany in the time period from 1991 to 1994 for example one would calculate: 
	\begin{equation} 
	Diff_{0-5,TotalExp}^{(9194, West)} = \hat{\beta}_{2}^{(9194, West)}*5 + \hat{\beta}_{3}^{(9194, West)} * 5^2
	\end{equation}

In this case $\hat{\beta}_{2}^{(9194, West)}$ would be the estimate for $
\beta_{2}$ from the first model specified in Section \ref{Sec:Method} using Pooled OLS on the sub sample of observations with survey years between 1991-1994 and sample region West Germany. This statistic  will be called \textit{Returns to Total Experience} henceforth. The values for New and Old Experience are calculated in an analogous way based on the Coefficients of the second model. Although education enters the wage regression only linearly and its effect could therefore be evaluated directly from the coefficients, it is evaluated in the same way as experience for the sake of simplicity and comparability. 
As seen in Section \ref{Sec:Data} however the sub samples differ significantly regarding the mean levels of each of the explanatory variable. On the one hand this will significantly affect the estimated returns as specified above, on the other it also makes it harder to judge the amount of human capital that is stored in each kind of experience and education for each of the sub samples. Therefore in the following results another statistic has been calculated, which can be explained as the mean value of the predicted contribution of each variable to the log wage. In the case of Total Experience and for the same sub sample as before this would be calculated as follows, where $I_{9194}^{west}$ is the set of all observations in that subset and $|I_{9194}^{west}|$ is the total number of those observations:
	\begin{equation} 
	\begin{split}
	Diff_{TotalExp}^{(9194, west)} =& \frac{1}{|I_{9194}^{west}|}\sum_{i \in 	I_{91-94}^{west}} \hat{\beta}_{2}^{(9194, west)}*TotalExp_{i}\\
 	&+\hat{\beta}_{3}^{(9194, west)} * TotalExp_{i}^2
	\end{split}
	\end{equation}
This value is then called \textit{Average Human Capital in Total Experience} in West Germany during the period of 1991 - 1994. Again the values for the other explanatory variables would be calculated analogously. 
\subsection{Experience}
From Figure \ref{fig:DiffComparisonExp} one can draw several observations regarding the development of returns to experience for the different sample regions. First of all and most obviously one can see that the returns for all types of experience in the West constantly exceed those in the East. Also the results seem to confirm the hypothesis regarding Old Experience gained from the Literature and the Analysis in Section \ref{Sec:Data}, namely that Old Experience is of no positive value in East Germany, but starts off with positive value in the West which then converges towards zero. Also can observe that the returns to New Experience converge toward those of Total Experience for both sample regions. As one might have expected the returns to Total Experience lie somewhere between those of New and Old Experience for each of the sub samples.
Based on the results in Figure \ref{fig:HumanCapitalExp} one can see that the trends observed for the returns to Old Experience are even more clearly visible regarding the Human Capital in Old Experience which is quite high for West Germany immediately after reunification but then converges towards zero. Due to the declining amount of Old Experience present in the sample the fluctuations and negative values observed for the returns in the later periods are somewhat smoothed and less visible with regard to human capital.
However regarding New Experience the rising amount of this kind of experience in the samples more than compensates the falling returns observed above and lead to slightly rising trend of Human Capital stored in New Experience. On the first glance it might be surprising that the amount of human capital stored in New Experience exceeds the value for total experience, however it is important to keep in mind that these are based on two different models and although $TotalExp = NewExp + OldExp$ holds, the same cannot be said for the estimated returns and human capital values. One might also observe from this graph that the relative Human Capital stored in New Experience for West Germans exceeds that for their East German counterparts by a relative constant margin, even though the amount of New Experience is actually higher in the East German sample in later periods.



\subsection{Education}
From Figure \ref{fig:DiffComparisonEdu} we can see that the differences in returns between different kinds of education and sample regions are much smaller than in the case of experience. One can see a steep increase in the returns to New Education, however one should bare in mind that there is virtually no New Education in the earlier sub samples. When looking at later periods where mean levels of the different kind of education are relatively similar one can see very similar values for Old, New and Total Education. This suggests that there is no notable difference in the value of Old and New Education in either of the sample regions. One can also see a rising trend in the returns to all types of education. This trend is steeper in East Germany leading to East German relative returns to education exceeding those in the West in the later samples. Regarding the relative Human Capital stored in the different types of Education one can see in Figure \ref{fig:HumanCapitalEdu} a positive trend for Total Education throughout the time frame, again at a steeper rate in the East than in the West resulting in higher levels in the East for later Periods. Regarding the different types of Education one can see not surprisingly a steeply rising share of human capital  in New Education a trend more pronounced in the West German sample where the human capital in New Education is higher throughout the time frame. 

