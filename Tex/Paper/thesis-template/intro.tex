\section{Introduction} \label{Sec:Intro}
\subsection{Motivation}
From the perspective of an labour economist the transition of countries from planned to market based economies in the early nineteen-nineties offers both a variety of interesting research questions with high policy relevance as well as the chance to evaluate the validity of various assumptions regarding labour markets.
The abruptness of the political change and the existence of a benchmark economy with the same institutional framework let the territory of the former German Democratic Republic stand out among other transitional economies in Europe. These aspects also enable analytical approaches that are not applicable in any other transitional economy.
The economical as well as the political importance of a successful integration of East Germany into the Economy of the Federal Republic generates many research problems with high policy relevance. This is especially true for the area of labour economics since differences in the wage levels as well as unemployment rate are often one of the main sources of public discontent regarding the economic development of East Germany. Persisting differences in both of these measures between East and West Germany show, that these topics remain relevant even a quarter of a century after reunification.  
\subsection{Problem Definition}
The original problem was researching the effect of pre-unification Education and Labor Market History in the GDR on the performance in the post-unification Labor Market. Based on the intention to analyse the dynamics of labour markets on a longer time frame including more recent data the decision was made to limit the focus on research questions that could be analysed using the data from the German Socio Economic Panel (GSOEP). Although this is a very rich panel dataset it includes less detailed information regarding the contents of employment than the BIBB-IAB survey mentioned in the title. From this decision it was therefore clear that the Labor Market History in the GDR would be mainly captured by the amount of experience gathered pre unification. Inspired by (\cite{gathmann_understanding_2004}) the focus was narrowed down to analyse the changing returns to education and experience from 1991 to 2014. Regarding this three research questions emerged to guide the analysis:

\begin{enumerate}
		\item \textit{How do returns to education and experience differ in East and West Germany ?} 
	\item \textit{How do these differences develop over time ?}
	\item \textit{How do these differences behave when differentiating between Experience and Education obtained pre- / post-unification?}
\end{enumerate}

Originally these questions were also analysed separately for different age and skill groups. However in the interest of brevity and clarity these aspects were not included in this report. 


\subsection{Literature Overview}
The uniqueness of the East German Labour market described above as well as the availability of high quality datasets have sparked a significant amount of research and publication on the German Labour market during the transition. A lot of these studies also make use of the GSOEP data.
Especially the first two the research questions have been the focus of many scientific papers in this area, most of which however are focused on the returns to experience. These studies have shown much flatter wage profiles across age  (\cite{krueger_comparative_1992},\cite{burda_getting_1997}) as well as experience (\cite{jurajda_when_2007}) in East Germany both before and in the first years after reunification, although the East German wage structure has changed significantly in the first years after reunification to resemble the West German structure more closely. (\cite{krueger_comparative_1992} shows that some of these differences especially regarding the experience profile persist for employees migrating from East to West Germany.
One study concludes that "... ,while endowments of education and training are comparable if not favourable in the East, returns to age were depressed under socialism and continue to be so several years after market relations were introduced."(\cite{burda_getting_1997}
, p. 24) suggesting very different findings regarding these research questions for experience and education. It has been shown that even after extending the analysis well into the twenty-first century and applying non-OLS estimators to account for endogeneity some of these differences persist (\cite{orlowski_east_2009}).
As mentioned above the main inspiration in the design of this study has been (\cite{gathmann_understanding_2004}), where Experience is separated into those years of experience gathered before reunification and those gathered after. In this paper it is found "... that socialist labor market experience has lost its economic value in the post-unification
labor market"(\cite{gathmann_understanding_2004}, p. 13).
One might summarise some of these results using the following hypotheses Regarding each of the research questions posed above:

\begin{enumerate}
	\item Returns to Education are similar in the East and West whereas returns to experience differ more widely. 
	\item Whereas returns show signs of convergence between East and West in the early years after reunification this trend slows and significant differences remain even decades later.
	\item Experience gathered in the GDR is of no value in the post reunification labour market.
\end{enumerate}

In Addition to testing these hypotheses this study will extend the literature in several ways. First of all the analysis will be extended to the whole time frame post reunification for which the GSOEP data is available (1991 - 2014). Instead of the binary comparison between pre- and post reunification which is the focus of most papers (especially early ones) this study will therefore be focused on the dynamic behaviour of wage structures post reunification. The approach of differentiating between New and Old Experience (\cite{gathmann_understanding_2004}) will be extended to Education and also applied to West German data to enable relative comparison. 


